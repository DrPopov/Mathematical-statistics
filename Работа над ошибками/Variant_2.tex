
\documentclass[a4paper,12pt, oneside]{book}

%  Русский язык
\usepackage{cmap}					% Улучшенный поиск русских слов
\usepackage[T2A]{fontenc}			% кодировка
\usepackage[utf8]{inputenc}			% кодировка исходного текста
\usepackage[english,russian]{babel}	% локализация и переносы
\usepackage{pscyr}					% нормальные шрифты

% Колонтитулы
\usepackage{fancybox,fancyhdr}
\usepackage{lastpage}
\fancyhf{}
\fancypagestyle{all}
{
	\fancyhead{}
	\fancyhead[C]{\vspace{-5mm}Контрольная работа 12 Вариант Попов Юрий СКБ-171\vspace{3mm}}
	\fancyfoot{}
	\fancyfoot[C]{\hfill  \thepage  \hfill}
}

% Картинки и графики
\usepackage{wrapfig}
\usepackage{pgfplots}
\pgfplotsset{compat=1.9}

% Ссылки
\usepackage{xcolor}
\usepackage{color,colortbl} % раскраска таблиц
\usepackage[unicode, pdftex]{hyperref}
\definecolor{linkcolor}{HTML}{000000} 	% цвет ссылок
\definecolor{urlcolor}{HTML}{8B000F}	% цвет гиперссылок
\hypersetup{urlcolor=urlcolor, linkcolor=linkcolor, colorlinks=true}

% Размеры
\setlength{\paperwidth}{210mm}
\setlength{\paperheight}{297mm}
\setlength{\textheight}{225mm} 			% высота без колонтитулов
\setlength{\textwidth}{170mm}			% ширина текста
\oddsidemargin=0pt
\setlength{\headheight}{2mm}			% высота колонтитула
\setlength{\headsep}{5mm} 	% от блока текста до верхнего колонтитула
\setlength{\footskip}{10mm}  % от блока текста до нижнего колонтитула

% Математика
\usepackage{amsmath,amsfonts,amssymb,amsthm,mathtools,mathtext} 
\usepackage{latexsym,array,epsfig,wasysym}
\usepackage[all]{xy}
\let\int\varint
\def\Int{\int\limits}
\def\IInt{\iint\limits}
\def\IIInt{\iiint\limits}
\DeclarePairedDelimiter\floor{\lfloor}{\rfloor}


% Оглавление 
\usepackage{setspace}

\usepackage{tocloft} %регулировка расположения TableOfContent (Оглавления) на странице

%\setcounter{tocdepth}{1} % отменяет вывод в оглавление subsection and subsubsection
%\setcounter{secnumdepth}{1} %отменяет номерацию секций в тексте и оглавлении.
\usepackage{tocloft} %регулировка расположения TableOfContent (Оглавления) на странице

\renewcommand{\cfttoctitlefont}{\hspace{0.38\textwidth} \bfseries\MakeUppercase} %уменьшаем размер шрифта и ровняем по центру

% % Межстрочные отступы в Оглавлении:
\setlength{\cftbeforetoctitleskip}{5mm} %отступ Оглавления от верхнего поля страницы.
\setlength{\cftbeforechapskip}{14mm} %отступ между главами
\setlength{\cftbeforesecskip}{5mm} %отступ между секциями \section{title}

% % Отступы от левого поля:
\setlength{\cftchapindent}{1mm} %отступ между левым полем и \chapter{}
\setlength{\cftsecindent}{13mm} %отступ между левым полем и \section{title}

% % Отточия в Оглавлении
\renewcommand\cftchapdotsep{\cftdot} %добавляет отточия после \chapter{title}
%\renewcommand{\cftchapleader}{\cftdotfill{\cftchapdotsep}} %делает отточия после \chapter{title} тонкими, (по умолчанию жирные).
\renewcommand\cftsecdotsep{\cftdot} %делает отточия после \section{title} частыми.

% % Интервалы между абзацами, главами и так далее:
\usepackage{titlesec}

\titleformat{\chapter}[display]
{\filcenter}
{\MakeUppercase{\chaptertitlename} \thechapter}
{8pt}
{\bfseries}{}

\titleformat{\section}
{\normalsize\bfseries}
{\thesection}
{1em}{}

\titleformat{\subsection}
{\normalsize\bfseries}
{\thesubsection}
{1em}{}

% Настройка вертикальных и горизонтальных отступов
\titlespacing*{\chapter}{0pt}{-30pt}{8pt}
\titlespacing*{\section}{\parindent}{*4}{*4}
\titlespacing*{\subsection}{\parindent}{*4}{*4}


\begin{document} % начало документа	
	\pagestyle{plain}
	
	\begin{titlepage}	
		\begin{center}
			{\Huge \textbf{Математическая статистика}}
			\vspace{30mm}
			
			{\Huge Работа над ошибками  № 1 \\}
			\vspace{30mm}
			
			{\huge Вариант №2}
			\vspace{30mm}
			
			{\Large Попов Юрий, СКБ-172}
		\end{center}
	\end{titlepage}
	
	
%---------------------------------------------------------------------------------Добавляем оглавление	
\begin{spacing}{0.99}          
	\tableofcontents               
\end{spacing}
\setcounter{secnumdepth}{-1} % убираем нумерацию 
%---------------------------------------------------------------------------------Добавляем оглавление	



%---------------------------------------------------------------------------------Предисловие
\newpage
\begin{center}
	{\Huge{\bf{Предисловие}}}
\end{center}



%---------------------------------------------------------------------------------Предисловие

%-------------------------------------------------------------------------------Задача 1
\begin{center}
	\chapter{Задача № 1}
\end{center}

Дана выборка объема n из "сдвинутого"$  $  экспоненциального распределения с плотностью распределения $ f(x) = \theta e^{-\theta(x-\theta)} , x \geq \theta, \theta > 0.$ Найти оценку для параметра $ \theta $ методом моментов.\\



Найдем первый момент:
$$
E = \int_{\theta}^{\infty} f(x) x dx = \int_{\theta}^{\infty} x \theta e^{-\theta(x-\theta)} dx = \frac{\theta^2 + 1}{\theta}
$$

$$
\frac{1}{n} \sum_{i=1}^{n} x_i =  \frac{\theta^2 + 1}{\theta}
$$

$$
\frac{\theta}{n} \sum_{i=1}^{n} x_i =  \theta^2 + 1
$$

$$
\theta^2 - \theta \frac{1}{n} \sum_{i=1}^{n} x_i +  1 = 0
$$

$$
D = \frac{1}{n} (\sum_{i=1}^{n} x_i)^2 - 4
$$

$$
\theta^2 - \theta \overline{x} + 1 = 0 
$$

$$
D = \overline{x^2} - 4
$$

$$
\theta_{1,2} = \frac{\overline{x} \pm \sqrt{\overline{x^2} -4}}{2}
$$

$$
\hat{\theta} = \frac{\overline{x} \mp \sqrt{\overline{x^2} - 4}}{2}
$$

Ответ: $ \hat{\theta} = \frac{\overline{x} \mp \sqrt{\overline{x^2} - 4}}{2}$
%-------------------------------------------------------------------------------Задача 1


%-------------------------------------------------------------------------------Задача 2
\begin{center}
	\chapter{Задача № 2}
\end{center}

Случайная величина $ \xi $ имеет распределение Вейбулла с плотностью распределения $f(x) = \theta x^{\theta - 1} e^{-x^{\theta}} , x \geq 0, \theta > 0.$ Над $ \xi $ проведено $ n $ наблюдений. Найти закон распределения случайной величины $ e^{X_{(1)}} $
%-------------------------------------------------------------------------------Задача 2

%-------------------------------------------------------------------------------Задача 3
\begin{center}
	\chapter{Задача № 3}
\end{center}
%-------------------------------------------------------------------------------Задача 3

%-------------------------------------------------------------------------------Задача 4
\begin{center}
	\chapter{Задача № 4}
\end{center}
%-------------------------------------------------------------------------------Задача 4


\begin{thebibliography}{99}
	\bibitem{rt1} 
	\bibitem{rt2} \href{https://towardsdatascience.com/what-is-exponential-distribution-7bdd08590e2a}{ссылка1}
	\bibitem{rt3}  \href{https://www.statisticshowto.datasciencecentral.com/exponential-distribution/}{ссылка2}
	\bibitem{rt4}  // \href{http://www.ams.jhu.edu/~dan/550.435/notes/COURSENOTES435.pdf}{ссылка3}
	\bibitem{rt5}  // \href{http://www.obzh.ru/nad/4-3.html}{ссылка4}
\end{thebibliography}

\end{document}
